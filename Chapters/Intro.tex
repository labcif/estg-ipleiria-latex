\addtocontents{toc}{\protect\vspace{\beforebibskip}} % Place slightly below the rest of the document content in the table

%************************************************
\chapter{Introdução}
\label{ch:introduction}
%************************************************


Este documento serve de orientação para o relatório da unidade curricular de Projecto Informático do Curso de Engenharia Informática da ESTG – IPLEIRIA. Como tal, é constituído por um conjunto predefinido de estilos a utilizar. Estes estilos devem ser utilizados sem serem alterados ou substituídos. Para começar facilmente a escrever o relatório, basta guardar uma cópia deste documento e substituir os campos e as secções de acordo com o projecto em questão.

Embora possa parecer uma abordagem demasiadamente descritiva para a escrita do relatório, as intenções pretendidas com este documento são:

\begin{itemize}
 \item Focar os alunos na produção de conteúdos com qualidade, em vez de se preocuparem com formatações de tipos de letra, parágrafos, etc.;
 \item Ao fornecer um documento de orientação de estilos a Escola beneficia de um aspecto profissional e consistente da globalidade dos seus relatórios de projecto.
\end{itemize}


Quanto ao conteúdo de uma introdução, ele deve preparar o leitor para o resto do relatório. Deve conter o detalhe suficiente para que alguém das áreas de conhecimento envolvidas possa entender o assunto do trabalho. A maior parte das introduções contêm três partes para fornecer contexto ao trabalho: objectivos, âmbito e background do trabalho do projecto. Estas partes muitas vezes sobrepõem-se, e podem por vezes ser omitidas simplesmente porque não faz sentido incluir alguma delas.

É de extrema importância considerar os objectivos do trabalho e do relatório na introdução. Se os autores não entenderem bem os objectivos do trabalho, dificilmente o leitor os entenderá. As seguintes questões ajudam a pensar nos objectivos do trabalho e na razão da escrita do relatório:

\begin{enumerate}
 \item O que foi descoberto ou provado?
 \item Em que tipos de problemas se trabalhou?
 \item Porque é que se trabalhou nestes problemas? Se o problema lhe foi atribuído, deve tentar-se saber as razões pelas quais os orientadores o formularam, e o que era suposto que os alunos aprendessem ao trabalharem neste problema;
 \item Qual a razão da escrita deste relatório?
 \item O que é que o leitor deve ficar a saber quando acabar de ler este relatório?
\end{enumerate}


O âmbito deve indicar as áreas de conhecimento envolvidas e realçar a metodologia utilizada no trabalho de projecto. Referir o âmbito do projecto na introdução ajuda o leitor a perceber os parâmetros de entrada do trabalho e do relatório, bem como a identificar as principais restrições consideradas (por exemplo “existem 5 Sistemas Operativos para trabalhar com determinado hardware, mas somente 3 foram considerados neste estudo”). As seguintes questões ajudam a pensar no âmbito do trabalho e do relatório:

\begin{enumerate}
 \item De que forma foi abordado o problema, e qual a razão para tal abordagem?
 \item Existiam outras abordagens óbvias que se poderiam ter adoptado ? Que limitações impediram que se tentassem outras abordagens?
 \item Que factores contribuíram para a escolha da forma de como se abordou o problema, e qual o mais relevante nessa escolha?
\end{enumerate}

A informação de background inclui os conhecimentos que o leitor deve possuir por forma a compreender o trabalho de projecto e correspondente relatório. Estes conhecimentos incluem a percepção de trabalhos prévios que motivaram a proposta do projecto corrente, ou referências a trabalhos teóricos e práticos relacionados com os objectivos e âmbito descritos acima. Devem remeter-se para anexos documentos que poderão ajudar na percepção de teorias, metodologias, técnicas ou ferramentas utilizadas no trabalho de projecto. As seguintes questões ajudam a pensar no background necessário para o trabalho e para o relatório:

\begin{enumerate}
 \item Que factos deve o leitor conhecer para perceber o relatório?
 \item Porque é que o projecto foi autorizado ou atribuído?
 \item Quem já fez trabalho prévio para resolver o problema colocado pelo projecto?
\end{enumerate}

Por fim, a introdução deve descrever como foi organizado o relatório, referindo brevemente o propósito de cada secção considerada no mesmo.

O resto deste documento dá uma breve perspectiva das partes seguintes que devem constar do relatório, bem como de outros aspectos de formatação.
